\documentclass[11pt]{article}
\usepackage[margin=1in]{geometry}
\usepackage{listings}
\usepackage{sectsty}

\title{\textbf{Specification for sensor unit serial protocol}}
\date{\vspace{-24pt} Revised 15 Apr 2015}

\begin{document}

\maketitle

%\parindent 0pt
\parskip   0pt

\section{General}
Two types of messages are defined for the sensor unit protocol. Control messages are sent by the controller to the sensor unit, while data messages are sent by the sensor unit to the controller. Data messages should not be sent except when requested by a control message. The sensor unit must be prepared to accept a control message at any time. All multi-byte fields described herein should be interpreted as little-endian. Each individual byte is big-endian, and all fields should be expected in the order they are listed. The UART should be configured at 9600/8-N-1 with no flow control.

\section{Control messages}
All control messages are three bytes; each consists of a 4-bit \textbf{CODE}, an 8-bit payload, and an 8-bit checksum:
\begin{verbatim}
[ CODE (4b) | resv (4b) ] [ Payload (8b)          ] [ Checksum ]
\end{verbatim}
Segments labeled \textbf{resv} are to be considered reserved space. Currently there are four \textbf{CODE}s defined, each having a different purpose.

Each checksum is calculated such that the sum of all three bytes (truncated to one byte) equals \textbf{0xFF}; i.e., \texttt{checksum = 0xFF - ((byte1 + byte2) \& 0xFF)}.

\subsection{Data request (0x1)}
This code indicates that the sensor unit should send data to the controller. One minute of data is buffered on the sensor unit; this data may be 		requested all at once or in four-sample chunks.
\begin{verbatim}
[ 0x1       | resv (4b) ] [ ADDR (4b) | resv (4b) ] [ Checksum ]
\end{verbatim}
If the \textbf{ADDR} field equals \textbf{0xF} (i.e., all bits set), then the entire last minute of data should be sent. If \textbf{ADDR} is less than \textbf{0xF}, then it is an index to a four-sample chunk of data. For example, if \textbf{ADDR} is \textbf{0x0}, the first four samples are sent; if \textbf{ADDR} is \textbf{0x1}, the second four samples are sent.

\subsection{Control set (0x2)}
This code sets the state of the control options available on the sensor unit. Respectively, these options are occupancy sensor trigger (\textbf{O}), temperature range trigger (\textbf{T}), manual relay control (\textbf{X}), and default relay state (\textbf{E}).
\begin{verbatim}
[ 0x2       | resv (4b) ] [ O T X E   | resv (4b) ] [ Checksum ]
\end{verbatim}
If \textbf{O} is set, then the occupancy sensor should be used to trigger the relay. If \textbf{T} is set, then the temperature range should be used to trigger the relay. If \textbf{X} is set, then the relay should be manually controlled according to the state of the \textbf{E} bit. The \textbf{E} bit should be used as the default relay state on sensor unit reset regardless of whether \textbf{X} is set. These options are controlled by one bit each.

\subsection{Temperature range set (0x3)}
This code is used to set the high and low thresholds for the temperature-based relay control option.
\begin{verbatim}
[ 0x3       | resv (4b) ] [ HILO (1b) | TEMP (7b) ] [ Checksum ]
\end{verbatim}
\textbf{HILO} controls whether the given \textbf{TEMP} is to be used as the high or low end of the range. A value of \textbf{0} indicates high, while \textbf{1} indicates low.

\subsection{Time set or reset request (0x4)}
This code can be used either to indicate that the sensor module should reset, or to set its internal clock.
\begin{verbatim}
[ 0x4       | resv (4b) ] [ FLAG (1b) | TIME (7b) ] [ Checksum ]
\end{verbatim}
If \textbf{FLAG} is 0, then the sensor unit should reset. Upon reset, the unit should send a data frame with the \textbf{0x80} bit set (as specified in \ref{subsec:errorfield}). This message will be replied to with another control message; the \textbf{FLAG} will be set and the \textbf{TIME} value will indicate the value from which the module should start counting.

\section{Data messages}
Each data message consists of \textbf{16} bytes and corresponds to one second of measurements. The first \textbf{12} bytes are payload; the next \textbf{1} byte is reserved; the next \textbf{1} byte is the \textbf{ERROR} flag mask; the final \textbf{2} bytes are a 16-bit CRC.

\vspace{12pt}
\noindent
\begin{minipage}{.5\textwidth}
\subsection{FLAGS field}
The \textbf{FLAGS} field (byte 0) is a status mask with the following indicator bits defined:
\begin{description}
	\itemsep 0pt
	\item[0x01] Occupancy status
	\item[0x02] AC (relay) status 
\end{description}
The remaining bits are reserved.

\subsection{ERROR field}
\label{subsec:errorfield}
The \textbf{ERROR} (byte 13) field is a status mask which provides indications of hardware failures. The bits are indicators defined as follows:
\begin{description}
	\itemsep 0pt
	\item[0x80] Reset complete / time not set
	\item[0x40] Relay failure
	\item[0x20] Bus failure
\end{description}
The remaining bits are reserved. 
\end{minipage}
\hspace{24pt}
\begin{minipage}{.25\textwidth}
\vspace{12pt}
\begin{verbatim}
 0 [_FLAGS_______]
 1 [ VOLTAGE     ]
 2 [_____________]
 3 [ CURRENT     ]
 4 [_____________]
 5 [ PHASE       ]
 6 [_____________]
 7 [_TEMPERATURE_]
 8 [ TIMESTAMP   ]
 9 [             ]
10 [             ]
11 [_____________]
12 [_RESERVED____]
13 [_ERROR_______]
14 [ CRC-16      ]
15 [_____________]
\end{verbatim}
\end{minipage}
\end{document}
